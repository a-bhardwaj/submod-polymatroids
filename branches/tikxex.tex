\documentclass{article}
\usepackage{tikz}
\begin{document}
\begin{tikzpicture}
    \begin{scope}[thick,font=\scriptsize]
    % Axes:
    % Are simply drawn using line with the `->` option to make them arrows:
    % The main labels of the axes can be places using `node`s:
    \draw [->] (0,0) -- (14,0) node [above left]  {$a$};
    \draw [->] (0,0) -- (0,7) node [below right] {$\psi(a)$};
    \draw[dashed, gray] (1,6.75) -- (1.8,6.75);
    \draw (0.43,6.2) node {$\Lambda(a)$};
    \draw[line width=1pt, lightgray] (1,6.2) -- (1.8,6.2);

    % Axes labels:
    % Are drawn using small lines and labeled with `node`s. The placement can be set using options
    \iffalse% Single
    % If you only want a single label per axis side:
    \draw (1,-3pt) -- (1,3pt)   node [above] {$1$};
    \draw (-3pt,1) -- (3pt,1)   node [right] {$i$};
    \else% Multiple
    % If you want labels at every unit step:
    \foreach \n in {0,1,2,3}{%
        \draw (2.5*\n,-1pt) -- (2.5*\n,1pt)   node [below] {$\bar{A}_{\n}$};
    }

    \draw[line width=1pt, lightgray] (0,0) -- (1.5,0);
    \draw (11,-1pt) -- (11,1pt)   node [below] {$\bar{A}_{m} - \bar{\delta}$};
    \draw[line width=1pt, lightgray] (9,4) -- (10,4);
    \draw[line width=1pt, lightgray, ->] (11,6) -- (13.5,6);
    \draw (12,-1pt) -- (12,1pt)   node [below] {$\bar{A}_{m}$};
    \foreach \n in {1,2,3}{%
        \draw (2.5*\n-1,-1pt) -- (2.5*\n-1,1pt)   node [below] {$\bar{A}_{\n} - \bar{\delta}$};
        \draw[thick,fill=black] (2.5*\n-1,\n) circle (1.5pt);
        \draw[thick,fill=white] (2.5*\n-1,\n-1) circle (1.5pt);
        \draw[line width=1pt, lightgray] (2.5*\n - 1,\n) -- (2.5*\n + 1.5,\n);
    }
    \foreach \n in {0,1,2,3}{%
        \draw[dashed,gray](2.5*\n,\n) -- (2.5*\n + 1.5,\n+1);
        \draw[dashed,gray](2.5*\n + 1.5,\n+1) -- (2.5*\n + 2.5,\n+1);
    }

    \draw	(10.3,5.4) node{$\cdot$};
    \draw	(10.3,5.1) node{$\cdot$};
    \draw	(10.3,4.8) node{$\cdot$};
    \draw	(10.3,4.5) node{$\cdot$};

    \draw[thick,fill=black] (9,4) circle (1.5pt);
    \draw[thick,fill=white] (9,3) circle (1.5pt);
    \draw[thick,fill=black] (11,6) circle (1.5pt);

    \draw[dashed,gray](10.7,5.8) -- (11,6);
    \draw[dashed,gray](11,6) -- (13.5,6);
    \fi
    \draw	(8,-4pt) node{$\cdot$};
    \draw	(8.5,-4pt) node{$\cdot$};
    \draw	(9,-4pt) node{$\cdot$};
    \draw	(9.5,-4pt) node{$\cdot$};
    \draw	(10,-4pt) node{$\cdot$};
    \end{scope}
%    % The circle is drawn with `(x,y) circle (radius)`
%    % You can draw the outer border and fill the inner area differently.
%    % Here I use gray, semitransparent filling to not cover the axes below the circle
%    \path [draw=none,fill=gray,semitransparent] (+1,-1) circle (3);
%    % Place the equation into the circle:
%    \node [below right,darkgray] at (+1,-1) {$|z-1+i| \leq 3$};
\end{tikzpicture}
\end{document} 